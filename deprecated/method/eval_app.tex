%Copyright (c) 2004 2005 2006 Atos Origin
%Permission is granted to copy, distribute and/or modify this
%document
%under the terms of the GNU Free Documentation License,
%      Version 1.2
%      or any later version published by the Free Software
%      Foundation;
%      with no Invariant Sections, no Front-Cover
%      Texts, and no Back-Cover
%      Texts.  A copy of the license is
%      included in the section entitled "GNU
%      Free Documentation License".
%
%$Id: eval_app.tex,v 1.5 2006/04/13 15:11:20 rsemeteys Exp $
\subsection{Evaluation sheet}
Every software release is described in an evaluation sheet. This document includes more detailed information 
than the identity card as it focuses on identifying, describing and analyzing in detail each 
evolution brought by the new release.
\subsubsection{Scoring}
Criteria are scored from 0 to 2. These scores will be used in Step 4 - "Selection" 
to compare and select software according to the weightings, representing the user's requirements 
specified in Step 3 - "Qualifcation".

The following paragraphs describe the criteria used for each axis of evaluation.
Note that the same criterion or similar criteria can appear on different axis.

\subsubsection{Functional coverage}
The functional grid is determined by the software's family and proceeds from the frame 
of reference of Step 1 - "Definition".
Consult the site \url{http://www.qsos.org} for details of functional grids by software families.

For each element of the grid, the scoring rule is as follows:


\begin{figure}
\center
\begin{tabular}{|c|c|c|}
\hline \TS{Functionality} & \TS{Score}\\
\hline Not covered & 0\\
\hline Partially covered & 1\\
\hline Completely covered & 2\\
\hline
\end{tabular}
\end{figure}


In certain cases it is necessary to use several functional grids for a same software, 
for instance when it belongs to more than one software family. 
In this case, the functional criteria are distributed on separated axis in order to be 
able to distinctly evaluate the functional coverage for each family.



\subsubsection{Risks from the user's perspective}

This axis of evaluation includes criteria to estimate risks incurred by the user when adopting 
free or open soure software.
Scoring of criteria is done independently of any particular user's context 
(the context is considered later in Step 3 - "Qualification").


Criteria are split into five categories:
\begin{itemize}
\item Intrinsic durability
\item Industrialized solution
\item Integration
\item Technical adaptability
\item Strategy
\end{itemize}


The following tables detail each one of these categories, by specifying the rule of notation 
to be used for each criterion.

%%%%%%%%%%%%%%%%%%%%%% Start Intrinsic durability
%%% Start Maturity
\begin{figure}
\center
\begin{tabular}{|p{2cm}|p{2cm}|p{2.8cm}|p{2.8cm}|p{2.8cm}|}
\hline \multicolumn{2}{|c|}{\TS{Intrinsic durability}} &
\multicolumn{3}{|c|}{\TS{Score}}\\
\cline{3-5} \multicolumn{2}{|c|}{} & \multicolumn{1}{|c|}{\TS{0}} &
\multicolumn{1}{|c|}{\TS{1}} &\multicolumn{1}{|c|}{\TS{2}}\\
\hline
\TS{Maturity}&
\TS{Age}&
For instance less than 3 months&
For instance between 3 months and 3 years &
For instance more than 3 years \\
\cline{2-5}&
\TS{Stability}&
Unstable software with numerous releases or patches generating side effects&
Stabilized production release existing but old. Difficulties to stabilize development releases&
Stabilized software. Releases provide bug fixes corrections but mainly new functionalities \\
\cline{2-5}&
\TS{History, known problems}&
Software knows several problems which can be prohibitive&
No known major problem or crisis&
History of good management of crisis situations\\
\cline{2-5}&
\TS{Fork probability, source of Forking}&
Software is very likely to be forked in the future&
Software comes from a fork but has very few chances of being forked in the future&
Software has very little chance of being forked. It does not come from a fork either\\
\hline
\end{tabular}
\end{figure}
%%% End Maturity

%%% Start Adoption 
\begin{figure}
\center
\begin{tabular}{|p{2cm}|p{2cm}|p{2.8cm}|p{2.8cm}|p{2.8cm}|}
\hline \multicolumn{2}{|c|}{\TS{Intrinsic durability}} &
\multicolumn{3}{|c|}{\TS{Score}}\\
\cline{3-5} \multicolumn{2}{|c|}{} & \multicolumn{1}{|c|}{\TS{0}} &
\multicolumn{1}{|c|}{\TS{1}} &\multicolumn{1}{|c|}{\TS{2}}\\
\hline
\TS{Adoption}&
\TS{Popularity (related to: general public, niche, ...)}&
Very few users identified&
Detectable use on Internet (sourceforge, freshmeat, google, ...)&
Numerous users, numerous references\\
\cline{2-5}&
\TS{References}&
None&
Few references, non critical usages&
Often implemented for critical applications\\
\cline{2-5}&
\TS{Contributing community}&
No community or without real activity (forum, mailing list,...)&
Existing community with a notable activity&
Strong community: big activity on forums, numerous contributors and advocates\\
\cline{2-5}&
\TS{Books}&
No book about the software&
Less than 5 books about the software are available&
More than 5 books about software are available, in several languages\\
\hline
\end{tabular}
\end{figure}
%%% End Adoption 

%%% Start Development leadership
\begin{figure}
\center
\begin{tabular}{|p{2cm}|p{2cm}|p{2.8cm}|p{2.8cm}|p{2.8cm}|}
\hline \multicolumn{2}{|c|}{\TS{Intrinsic durability}} &
\multicolumn{3}{|c|}{\TS{Score}}\\
\cline{3-5} \multicolumn{2}{|c|}{} & \multicolumn{1}{|c|}{\TS{0}} &
\multicolumn{1}{|c|}{\TS{1}} &\multicolumn{1}{|c|}{\TS{2}}\\
\hline
\TS{Development leadership}&
\TS{Leading team}&
1 to 2 individuals involved, not clearly identified&
Between 2 and 5 independent people&
More than 5 people\\
\cline{2-5}&
\TS{Management style}&
Complete dictatorship&
Enlightened despotism&
Council of architects with identified leader (e.g: ASF, ...)\\
\hline
\end{tabular}
\end{figure}
%%% End Development leadership

%%% Start Activity
\begin{figure}
\center
\begin{tabular}{|p{2cm}|p{2cm}|p{2.8cm}|p{2.8cm}|p{2.8cm}|}
\hline \multicolumn{2}{|c|}{\TS{Intrinsic durability}} &
\multicolumn{3}{|c|}{\TS{Score}}\\
\cline{3-5} \multicolumn{2}{|c|}{} & \multicolumn{1}{|c|}{\TS{0}} &
\multicolumn{1}{|c|}{\TS{1}} &\multicolumn{1}{|c|}{\TS{2}}\\
\hline
\TS{Activity}&
\TS{Developers, identification, turnover}&
Less than 3 developers, not clearly identified&
Between 4 and 7 developers, or more unidentified developers with important turnover&
More than 7 developers clearly identified, very stable team\\
\cline{2-5}&
\TS{Activity on bugs}&
Slow reactivity in forum or on mailing list, or nothing regarding bug fixes in release notes&
Detectable activity but without process clearly exposed, long reaction/resolution time&
Strong reactivity based on roles and tasks assignment\\

\cline{2-5}&
\TS{Activity on functionalities}&
No or few new functionalities&
Evolution of the product driven by the core team or by user's request without any clearly explained process&
Tool(s) to manage feature requests, strong interaction with roadmap\\
\cline{2-5}&
\TS{Activity on releases}&
Very weak activity on both production and development releases&
Activity on production and development releases. Frequent minor releases (bug fixes)&
Important activity with frequent minor releases (bugs fixes) and planned major 
releases relating to the roadmap forcast\\
\hline
\end{tabular}
\end{figure}
%%% End Activity


%%% Start Independence of developments 
\begin{figure}
\center
\begin{tabular}{|p{2cm}|p{2cm}|p{2.8cm}|p{2.8cm}|p{2.8cm}|}
\hline \multicolumn{2}{|c|}{\TS{Intrinsic durability}} &
\multicolumn{3}{|c|}{\TS{Score}}\\
\cline{3-5} \multicolumn{2}{|c|}{} & \multicolumn{1}{|c|}{\TS{0}} &
\multicolumn{1}{|c|}{\TS{1}} &\multicolumn{1}{|c|}{\TS{2}}\\
\hline
\TS{Independence of developments}&
\TS{Independence of developments}&
Developments realized at 100\% by employees of a single company&
60\% maximum&
20\% maximum\\
\hline
\end{tabular}
\end{figure}
%%% End Independence of developments

% Le clearpage impose a Latex de poser les objets flotant qu'il a en memoire
% ! LaTeX Error: Too many unprocessed floats.
% TODO : le must serait de ne pas utiliser de float
\clearpage
%%%%%%%%%%%%%%%%%%%%%% End Intrinsic durability

%%%%%%%%%%%%%%%%%%%%%% Start Industrialised solution
%%% Start Services 
\begin{figure}
\center
\begin{tabular}{|p{2cm}|p{2cm}|p{2.8cm}|p{2.8cm}|p{2.8cm}|}
\hline \multicolumn{2}{|c|}{\TS{Industrialised solution}} &
\multicolumn{3}{|c|}{\TS{Score}}\\
\cline{3-5} \multicolumn{2}{|c|}{} & \multicolumn{1}{|c|}{\TS{0}} &
\multicolumn{1}{|c|}{\TS{1}} &\multicolumn{1}{|c|}{\TS{2}}\\
\hline
\TS{Services}&
\TS{Training}&
No offer of training identified&
Offer exists but is restricted geographically and to one language or is provided by a single contractor&
Rich offers provided by several contractors, in several languages and split into modules of gradual levels\\
\cline{2-5}&
\TS{Support}&
No offer of support except via public forums and mailing lists&
Offer exists but is provided by a single contractor without strong commitment quality of services&
Multiple service providers with strong commitment (e.g: guaranteed resolution time)\\
\cline{2-5}&
\TS{Consulting}&
No offer of consulting service&
Offer exists but is restricted geographically and to one language or is provided by a single contractor&
Consulting services provided by different contractors in several languages\\
\hline
\end{tabular}
\end{figure}
%%% End Services 

%%% Start Documentation 
\begin{figure}
\center
\begin{tabular}{|p{2cm}|p{2cm}|p{2.8cm}|p{2.8cm}|p{2.8cm}|}
\hline \multicolumn{2}{|c|}{\TS{Industrialised solution}} &
\multicolumn{3}{|c|}\TS{{Score}}\\
\cline{3-5} \multicolumn{2}{|c|}{} & \multicolumn{1}{|c|}{\TS{0}} &
\multicolumn{1}{|c|}{\TS{1}} &\multicolumn{1}{|c|}{\TS{2}}\\
\hline
\TS{Documenta\-tion}&
\TS{Documenta\-tion}&
No user documentation&
Documentation exists but shifted in time, is restricted to one language or is poorly detailed&
Documentation always up to date, translated and possibly adapted to different target readers 
(end user, sysadmin, manager, ?)\\
\hline
\end{tabular}
\end{figure}
%%% End Documentation 

%%% Start Quality Assurance
\begin{figure}
\center
\begin{tabular}{|p{2cm}|p{2cm}|p{2.8cm}|p{2.8cm}|p{2.8cm}|}
\hline \multicolumn{2}{|c|}{\TS{Industrialised solution}} &
\multicolumn{3}{|c|}{\TS{Score}}\\
\cline{3-5} \multicolumn{2}{|c|}{} & \multicolumn{1}{|c|}{\TS{0}} &
\multicolumn{1}{|c|}{\TS{1}} &\multicolumn{1}{|c|}{\TS{2}}\\
\hline
\TS{Quality Assurance}&
\TS{Quality Assurance}&
No QA process&
Identifies QA process but not much formalized and with no tool&
Automatic testing process included in code's life-cycle with publication of results\\
\cline{2-5}&
\TS{Tools}&
No bug or feature request management tool&
Standard tools provided (for instance by a hosting forge) but poorly used&
Very active use of tools for roles/tasks allocation and progess monitoring\\
\hline
\end{tabular}
\end{figure}
%%% End Quality Assurance 


%%% Start Packaging - part 1
\begin{figure}
\center
\begin{tabular}{|p{2cm}|p{2cm}|p{2.8cm}|p{2.8cm}|p{2.8cm}|}
\hline \multicolumn{2}{|c|}{\TS{Industrialised solution}} &
\multicolumn{3}{|c|}{\TS{Score}}\\
\cline{3-5} \multicolumn{2}{|c|}{} & \multicolumn{1}{|c|}{\TS{0}} &
\multicolumn{1}{|c|}{\TS{1}} &\multicolumn{1}{|c|}{\TS{2}}\\
\hline
\TS{Packaging}&
\TS{Source}&
Software can't be installed from source without a lot of work&
Installation from source is limited and depends on very strict conditions (OS, arch,
lib, ...)&
Installation from source is easy\\
\cline{2-5}&
\TS{Debian}&
The software is not packaged for Debian&
A Debian package exists but it has important issues or it doesn't have
official support&
The software is packaged in the distribution\\
\cline{2-5}&
\TS{FreeBSD}&
The software is not packaged for FreeBSD&
A port exists but it has important issues or it doesn't have official
support&
A official port exists in FreeBSD\\
\cline{2-5}&
\TS{HP-UX}&
The software is not packaged for HP-UX&
A package exists but it has important issues or it doesn't have official
support&
A tested package is provided for HP-UX\\
\cline{2-5}&
\TS{MacOSX}&
The software is not packaged for MacOSX&
A package exists but it has important issues or it doesn't have official
support&
The software is packaged in the distribution\\
\cline{2-5}&
\TS{Mandriva}&
The software is not packaged for Mandriva&
A package exists but it has important issues or it doesn't have official
support&
The software is packaged in the distribution\\
\hline
\end{tabular}
\end{figure}
%%% End Packaging - part 1

%%% Start Packaging - part 2 
\begin{figure}
\center
\begin{tabular}{|p{2cm}|p{2cm}|p{2.8cm}|p{2.8cm}|p{2.8cm}|}
\hline \multicolumn{2}{|c|}{\TS{Industrialised solution}} &
\multicolumn{3}{|c|}{\TS{Score}}\\
\cline{3-5} \multicolumn{2}{|c|}{} & \multicolumn{1}{|c|}{\TS{0}} &
\multicolumn{1}{|c|}{\TS{1}} &\multicolumn{1}{|c|}{\TS{2}}\\
\hline
\TS{Packaging}&
\TS{NetBSD}&
The software is not packaged for NetBSD&
A port exists but it has important issues or it doesn't have official
support&
A official port exists in NetBSD\\
\cline{2-5}&
\TS{OpenBSD}&
The software is not packaged for OpenBSD&
A port exists but it has important issues or it doesn't have official
support&
A official port exists in OpenBSD\\
\cline{2-5}&
\TS{RedHat/Fedora}&
The software is not packaged for RedHat/Fedora&
A package exists but it has important issues or it doesn't have official
support&
The software is packaged in the distribution\\
\cline{2-5}&
\TS{Solaris}&
The software is not packaged for Solaris&
A package exists but it has important issues or it doesn't have official
support (e.g: SunFreeware.com)&
The software is supported by Sun for Solaris\\
\cline{2-5}&
\TS{SuSE}&
The software is not packaged for SuSE&
A package exists but it has important issues or it doesn't have official
support&
The software is packaged in the distribution\\
\cline{2-5}&
\TS{Windows}&
The project can't be installed on Windows&
A package exists but is limited or has important issues or covers only
specific Windows releases (e.g: Windows2000 and WindowsXP)&
Windows is fully supported and a package is provided\\
\hline
\end{tabular}
\end{figure}
%%% End Packaging - part 2


%%% Start Exploitability
\begin{figure}
\center
\begin{tabular}{|p{2cm}|p{2cm}|p{2.8cm}|p{2.8cm}|p{2.8cm}|}
\hline \multicolumn{2}{|c|}{\TS{Industrialised solution}} &
\multicolumn{3}{|c|}{\TS{Score}}\\
\cline{3-5} \multicolumn{2}{|c|}{} & \multicolumn{1}{|c|}{\TS{0}} &
\multicolumn{1}{|c|}{\TS{1}} &\multicolumn{1}{|c|}{\TS{2}}\\
\hline
\TS{Exploitability}&
\TS{Ease of use, ergonomics}&
Difficult to use, requires an in depth knowledge of the software functionality&
Austere and very technical ergonomics&
GUI including help functions and elaborated ergonomics (e.g: skins/themes management)\\
\cline{2-5}&
\TS{Administration / Monitoring}&
No administrative or monitoring functionalities&
Existing functionalities but incomplete and in need of improvement&
Complete and easy-to-use administration and monitoring functionalities. Possible integration with 
external tools (e.g: via SNMP, ...)\\
\hline
\end{tabular}
\end{figure}
%%% End Exploitability
%%%%%%%%%%%%%%%%%%%%%% End Industrialised solution

%%%%%%%%%%%%%%%%%%%%%% Start Technical adaptability
%%% Start Modularity
\begin{figure}
\center
\begin{tabular}{|p{2cm}|p{2cm}|p{2.8cm}|p{2.8cm}|p{2.8cm}|}
\hline \multicolumn{2}{|c|}{\TS{Technical adaptability}} &
\multicolumn{3}{|c|}{\TS{Score}}\\
\cline{3-5} \multicolumn{2}{|c|}{} & \multicolumn{1}{|c|}{\TS{0}} &
\multicolumn{1}{|c|}{\TS{1}} &\multicolumn{1}{|c|}{\TS{2}}\\
\hline
\TS{Modularity}&
\TS{Modularity}&
Monolithic software&
Presence of high level modules allowing a first level of software adaptation&
Modular conception, allowing easy adaptation of the software by selecting modules or even developing new ones\\
\hline
\end{tabular}
\end{figure}
%%% End Modularity

%%% Start By-products
\begin{figure}
\center
\begin{tabular}{|p{2cm}|p{2cm}|p{2.8cm}|p{2.8cm}|p{2.8cm}|}
\hline \multicolumn{2}{|c|}{\TS{Technical adaptability}} &
\multicolumn{3}{|c|}{\TS{Score}}\\
\cline{3-5} \multicolumn{2}{|c|}{} & \multicolumn{1}{|c|}{\TS{0}} &
\multicolumn{1}{|c|}{\TS{1}} &\multicolumn{1}{|c|}{\TS{2}}\\
\hline
\TS{By-products}&
\TS{Code modification}&
Everything by hand&
Recompilation possible but complex without any tools or documentation&
Recompilation with tools (e.g: make, ANT, ...) and documention provided\\
\cline{2-5}&
\TS{Code extension}&
Any modification requires code recompilation&
Architecture designed for static extension but requires recompilation&
Principle of plug-in, architecture designed for dynamic extension without recompilation\\
\hline
\end{tabular}
\end{figure}
%%% End By-products
%%%%%%%%%%%%%%%%%%%%%% End Technical adaptability

%%%%%%%%%%%%%%%%%%%%%% Start Strategy 
%%% Start License 
\begin{figure}
\center
\begin{tabular}{|p{2cm}|p{2cm}|p{2.8cm}|p{2.8cm}|p{2.8cm}|}
\hline \multicolumn{2}{|c|}{\TS{Strategy}} & \multicolumn{3}{|c|}{\TS{Score}}\\
\cline{3-5} \multicolumn{2}{|c|}{} & \multicolumn{1}{|c|}{\TS{0}} &
\multicolumn{1}{|c|}{\TS{1}} &\multicolumn{1}{|c|}{\TS{2}}\\
\hline
\TS{License}&
\TS{Permissiveness (to be weighted only if user wants to become owner of code)}&
Very strict license, like GPL&
Moderate permissive license located between both extremes (GPL and BSD), dual-licensing depending on 
the type of user (person, company, ...) or their activities&
Very permissive like BSD or Apache licenses\\
\cline{2-5}&
\TS{Protection against proprietary forks}&
Very permissive like BSD or Apache licenses&
Moderate permissive license located between both extremes (GPL and BSD), dual-licensing depending on 
the type of user (person, company, ...) or their activities&
Very strict license, like GPL\\
\hline
\end{tabular}
\end{figure}
%%% End License 

%%% Start Copyright owners
\begin{figure}
\center
\begin{tabular}{|p{2cm}|p{2cm}|p{2.8cm}|p{2.8cm}|p{2.8cm}|}
\hline \multicolumn{2}{|c|}{\TS{Strategy}} & \multicolumn{3}{|c|}{\TS{Score}}\\
\cline{3-5} \multicolumn{2}{|c|}{} & \multicolumn{1}{|c|}{\TS{0}} &
\multicolumn{1}{|c|}{\TS{1}} &\multicolumn{1}{|c|}{\TS{2}}\\
\hline
\TS{Copyright owners}&
\TS{Copyright owners}&
Rights held by a few individuals or entities, making it easier to change the license&
Rights held by numerous individuals owning the code in a homogeneous way, making relicensing very difficult&
Rights held by a legal entity in whom the community trusts (e.g. FSF or ASF)\\
\hline
\end{tabular}
\end{figure}
%%% End Copyright owners

%%% Start Modification of source code
\begin{figure}
\center
\begin{tabular}{|p{2cm}|p{2cm}|p{2.8cm}|p{2.8cm}|p{2.8cm}|}
\hline \multicolumn{2}{|c|}{\TS{Strategy}} & \multicolumn{3}{|c|}{\TS{Score}}\\
\cline{3-5} \multicolumn{2}{|c|}{} & \multicolumn{1}{|c|}{\TS{0}} &
\multicolumn{1}{|c|}{\TS{1}} &\multicolumn{1}{|c|}{\TS{2}}\\
\hline
\TS{Modification of source code}&
\TS{Modification of source code}&
No practical way to propose code modifications&
Tools provided to access and modify code (like CVS or SVN) but not really used to develop the software&
The code modification process is well defined, exposed and respected, based on roles assignment\\
\hline
\end{tabular}
\end{figure}
%%% End Modification of source code

%%% Start roadmap 
\begin{figure}
\center
\begin{tabular}{|p{2cm}|p{2cm}|p{2.8cm}|p{2.8cm}|p{2.8cm}|}
\hline \multicolumn{2}{|c|}{\TS{Strategy}} & \multicolumn{3}{|c|}{\TS{Score}}\\
\cline{3-5} \multicolumn{2}{|c|}{} & \multicolumn{1}{|c|}{\TS{0}} &
\multicolumn{1}{|c|}{\TS{1}} &\multicolumn{1}{|c|}{\TS{2}}\\
\hline
\TS{Roadmap}&
\TS{Roadmap}&
No published roadmap&
Existing roadmap without planning&
Versionned roadmap, with planning and measure of delays\\
\hline
\end{tabular}
\end{figure}
%%% End Roadmap

%%% Start Sponsor 
\begin{figure}
\center
\begin{tabular}{|p{2cm}|p{2cm}|p{2.8cm}|p{2.8cm}|p{2.8cm}|}
\hline \multicolumn{2}{|c|}{\TS{Strategy}} & \multicolumn{3}{|c|}{\TS{Score}}\\
\cline{3-5} \multicolumn{2}{|c|}{} & \multicolumn{1}{|c|}{\TS{0}} &
\multicolumn{1}{|c|}{\TS{1}} &\multicolumn{1}{|c|}{\TS{2}}\\
\hline
\TS{Sponsor}&
\TS{Sponsor}&
Software has no sponsor, the core team is not paid&
Software has an unique sponsor who might determine its strategy&
Software is sponsored by industry\\
\hline
\end{tabular}
\end{figure}
%%% Start Sponsor 

%%% Start Strategical independence
\begin{figure}
\center
\begin{tabular}{|p{2cm}|p{2cm}|p{2.8cm}|p{2.8cm}|p{2.8cm}|}
\hline \multicolumn{2}{|c|}{\TS{Strategy}} & \multicolumn{3}{|c|}{\TS{Score}}\\
\cline{3-5} \multicolumn{2}{|c|}{} & \multicolumn{1}{|c|}{\TS{0}} &
\multicolumn{1}{|c|}{\TS{1}} &\multicolumn{1}{|c|}{\TS{2}}\\
\hline
\TS{Strategical independence}&
\TS{Strategical independence}&
No detectable strategy or strong dependency on one unique actor (person, company, sponsor, ...)&
Strategical vision shared with several other free and open source projects but without strong 
commitment from copyrights owners&
Strong independence of the core team, legal entity holding rights, strong involvement in the standardization process\\
\hline
\end{tabular}
\end{figure}
%%% End Strategical independence
%%%%%%%%%%%%%%%%%%%%%% End Strategy 

\clearpage

\subsubsection{Risks from the service provider's perspective}
This axis of evaluation regroups criteria to estimate risks incurred by a contractor offering services around a free or open source software (expertise, intregration, development, support, ...). It is notably on this basis that its level of commitment can be determined.

%%%%%%%%%%%%%%%%%%%%%% Start Services provinding
%%% Start Maintenability
\begin{figure}
\center
\begin{tabular}{|p{2cm}|p{2cm}|p{2.8cm}|p{2.8cm}|p{2.8cm}|}
\hline \multicolumn{2}{|c|}{\TS{Services provinding}} & \multicolumn{3}{|c|}{\TS{Score}}\\
\cline{3-5} \multicolumn{2}{|c|}{} & \multicolumn{1}{|c|}{\TS{0}} &
\multicolumn{1}{|c|}{\TS{1}} &\multicolumn{1}{|c|}{\TS{2}}\\
\hline
\TS{Maintenability}&
\TS{Quality of source code}&
Not very readable code or of poor quality, incoherence in coding styles&
Readable code but not really commented in detail&
Readable and commented code, implementing classic design patterns with a coherent and applied coding policy\\
\cline{2-5}&
\TS{Technological dispersion}&
Use of numerous different languages&
One main language with certain modules coded in other languages for specific and limited requirements&
One unique language\\
\cline{2-5}&
\TS{Intrinsic complexity}&
Very complex code requiring high level of expertise to perform modifications without generating side effects&
Not very complex code but still requiring expertise in programming languages and software design&
Simple code and design, easy to modify\\
\cline{2-5}&
\TS{Technical documentation}&
No documentation (development guide or automatically generated doc like javadoc)&
Incomplete or old documentation without conception and architectural considerations&
Detailed and up to date documentation, including conception, architecture design and coding considerations\\
\hline
\end{tabular}
\end{figure}
%%% Start Maintenability

%%% Start Code mastery
\begin{figure}
\center
\begin{tabular}{|p{2cm}|p{2cm}|p{2.8cm}|p{2.8cm}|p{2.8cm}|}
\hline \multicolumn{2}{|c|}{\TS{Services provinding}} & \multicolumn{3}{|c|}{\TS{Score}}\\
\cline{3-5} \multicolumn{2}{|c|}{} & \multicolumn{1}{|c|}{\TS{0}} &
\multicolumn{1}{|c|}{\TS{1}} &\multicolumn{1}{|c|}{\TS{2}}\\
\hline
\TS{Code mastery}&
\TS{Direct}&
No direct expertise of the source code&
Mastery of code but limited to a single person or to only a part of the source&
Several individuals mastering the code and covering together the totality of the source\\
\cline{2-5}&
\TS{Indirect}&
No indirect expertise on the source code&
Strong mastery via external expertise provided by partners&
Partnership with the copyrights owner and/or the core team\\
\hline
\end{tabular}
\end{figure}
%%% End Code mastery
%%%%%%%%%%%%%%%%%%%%%% End Services provinding
\clearpage


\subsubsection{Granularity of scores}
As stated above, it is possible to iterate the QSOS process. 
At the evaluation step this brings the capacity to score criteria in 
three passes with different levels of granularity:

\begin{itemize}
\item First the five main categories
\item Then the sub categories of each category
\item Finally every remaining criterion
\end{itemize}


The general process is thus not hindered if not all of the scored criteria are available.


Once all criteria have been scored, the score of the firsts two levels is calculated by the weighed average of scores of the direcly inferior level.

\subsection{O3S tool}
The O3S tool allows the entry of raw data and the evaluation of software on the three majors axis, 
as well as generation of the identidy cards of evaluated software.


The granularity of evaluation is managed as follows: as long as all
criteria composing a sub category are not scored, its score is not 
calculated but entered by the user. As soon as all criteria are scored, 
its score is then automatically calculated.
