%Copyright (c) 2004 2005 2006 Atos Origin
%Permission is granted to copy, distribute and/or modify this
%document
%under the terms of the GNU Free Documentation License,
%      Version 1.2
%      or any later version published by the Free Software
%      Foundation;
%      with no Invariant Sections, no Front-Cover
%      Texts, and no Back-Cover
%      Texts.  A copy of the license is
%      included in the section entitled "GNU
%      Free Documentation License".
%
%$Id: manifest.tex,v 1.2 2006/03/07 16:35:51 rsemeteys Exp $
\section{The QSOS Manifesto}

\subsection{Why a method?}
For a company, the choice to opt for software as a component of its information
system, whether this software is Open Source or commercial, rests on the
analysis of needs and constraints (technical, functional and strategic) and on the adequacy 
of the software to these needs and constraints.

However, when one plans to study the adequacy of open source software, it is necessary to have 
a method of qualification and selection adapted to characteristics of this type of software. 
It is, for instance, particularly important to precisely examine the constraints and risks 
specific to open source software. 
Since the open source field is very rich and has a very broad scope, it is also 
necessary to use a qualification method allowing to
differentiate the quite often numerous candidates to meet both technical, functionnal and strategic requirements.

Beside "natural" questions like:
\begin{itemize}
\item Which software meets best my actual/planned {\bf technical} requirements?
\item Which software meets best my actual/planned {\bf functional} requirements?
\end{itemize}

Here follow some questions every company should answer before making any decision:
\begin{itemize}
\item What is the durability of the software? What are the risks of Forks? How to anticipate and manage them?
\item What level of stability to expect? How to manage dysfunctions?
\item What is the expected and available support level provided on the software?
\item Is it possible to influence further development of the software (addition of new or specific functionalities)?
\end{itemize}


To be able to answer serenely these types of questions and thus set up an
efficient risk management process, it is imperative to have a method allowing:
\begin{itemize}
\item software qualification by integrating the open source characteristics;
\item software comparisons according to formalized needs requirements of weighted criteria, in order to make a final choice.
\end{itemize}

These are the various points which made Atos Origin conceive and formalize the method for Qualification and Selection of Open Source software (QSOS).


\subsection{Why a free method?}
For us, such a method as well as the results it generates, must be made available to all under the terms of a free licence. Only such a licence is capable of ensuring the promotion of the open source movement, via in particular:
\begin{itemize}
\item the ability for all to re-use already available work for qualification and evaluation;
\item the quality and objectivity of documents generated, always perfected according to principles of transparency and
peer reviews.
\end{itemize}
For these reasons Atos Origin decided to make available the QSOS method, and the documents generated during its application (functional grids, identity cards and evaluation sheets), under the terms of the GNU Free Documentation License.
