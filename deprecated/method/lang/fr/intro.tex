%Copyright (c) 2004 2005 2006 Atos Origin
%Permission is granted to copy, distribute and/or modify this
%document
%under the terms of the GNU Free Documentation License,
%      Version 1.2
%      or any later version published by the Free Software
%      Foundation;
%      with no Invariant Sections, no Front-Cover
%      Texts, and no Back-Cover
%      Texts.  A copy of the license is
%      included in the section entitled "GNU
%      Free Documentation License".
%
%$Id: intro.tex,v 1.1 2006/02/16 18:21:09 goneri Exp $
\section{Introduction}

\subsection{Objet du document}
Ce document pr�sente la m�thode, baptis�e << QSOS >> (Qualification et S�lection de logiciels Open Source), con�ue par Atos Origin pour qualifier et s�lectionner les logiciels Open Source dans le cadre de ses travaux de support et de veille technologique.


La m�thode peut s'int�grer dans le cadre plus g�n�ral d'un processus de veille technologique qui n'est pas pr�sent� ici, et d�crit un processus de constitution des fiches d'identit� et d'�valuation de logiciels libres.

\subsection{Public vis�}
Le pr�sent document vise les publics suivants :
\begin{itemize}
\item les personnes curieuses de se documenter sur la m�thode � titre professionnel comme personnel ;
\item les communaut�s des projets Open Source ;
\item les experts du secteur informatique d�sirant conna�tre et appliquer la m�thode dans leur travail quotidien d'�valuation et de s�lection de composants dans l'optique de b�tir des solutions logicielles r�pondant � leurs besoins ou � ceux de leurs clients.
\end{itemize}

