%Copyright (c) 2004 2005 2006 Atos Origin
%Permission is granted to copy, distribute and/or modify this
%document
%under the terms of the GNU Free Documentation License,
%      Version 1.2
%      or any later version published by the Free Software
%      Foundation;
%      with no Invariant Sections, no Front-Cover
%      Texts, and no Back-Cover
%      Texts.  A copy of the license is
%      included in the section entitled "GNU
%      Free Documentation License".
%
%$Id: intro.tex,v 1.2 2006/03/07 16:35:51 rsemeteys Exp $
\section{Introduction}

\subsection{Document objective}
This document describes the QSOS method (Qualification and Selection of software Open Source), conceived by Atos Origin to qualify and select the Free and Open Source software as a basis of its support and technological survey services.

The method can be integrated within a more general process of technological watch which is not presented here. It describes a process to set up identity cards and evaluation sheets for free and open source software.

\subsection{Targeted readers}
This documlent targets the following readers:
\begin{itemize}
\item people curious to know more about the method, on a professional or personal basis;
\item communities of Free and Open Source projects;
\item IT experts wishing to know and use the method in evaluating and selecting components to build systems either for their own needs or for their customers.
\end{itemize}


