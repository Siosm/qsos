%Copyright (c) 2004 2005 2006 Atos Origin
%Permission is granted to copy, distribute and/or modify this
%document
%under the terms of the GNU Free Documentation License,
%      Version 1.2
%      or any later version published by the Free Software
%      Foundation;
%      with no Invariant Sections, no Front-Cover
%      Texts, and no Back-Cover
%      Texts.  A copy of the license is
%      included in the section entitled "GNU
%      Free Documentation License".
%
%$Id: glossary.tex,v 1.1 2006/02/16 18:21:09 goneri Exp $
\section*{Glossaire}
%\addcontentsline{toc}{chapter}{Glossaire}


\paragraph{Fork}
Un fork est un �v�nement qui appara�t parfois dans le d�veloppement
d'un projet informatique, typiquement dans des projets communautaires (cas de beaucoup
de logiciels libres), quand les opinions au sein de l'�quipe de d�veloppement divergent
sur le chemin � prendre et qu'elles ne sont pas conciliables. Le d�veloppement du
logiciel part alors dans deux directions diff�rentes, sous l'impulsion des deux camps.

\paragraph{M�thode QSOS}
M�thode de Qualification et de S�lection de logiciels Open Source, con�ue et utilis�e
par Atos Origin pour ses travaux de support et de veille technologique. Elle est mise
� disposition - sous licence libre - sur le site \url{http://www.qsos.org}.

\paragraph{O3S}
Open Source Selection Software. Outil informatique d�velopp� par Atos Origin impl�mentant la m�thode QSOS, qui sera utilis� sur le site \url{http://www.qsos.org} pour cr�er, modifier et visualiser les fiches d'identit� et d'�valuation.

\paragraph{Prestataire de services}
Toute entreprise d�sireuse d'offrir des services autour de logiciels libres ou Open Source (expertise, int�gration, d�veloppement, support, ...).

\paragraph{Utilisateur}
Toute personne physique, entit�, entreprise ou administration utilisatrice ou future utilisatrice de logiciels libres ou Open Source.
